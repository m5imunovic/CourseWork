\section*{Zaključak}
\label{ch:conc}
\addcontentsline{toc}{section}{Zaključak}

Razvijeni model pokazuje kako stabla odlučivanja omogućuju kreiranje vrlo uspješnih modela s vrlo malo ili nimalo domenskoga znanja. Sami algoritam u sebi ima heuristiku koja mu omogućuje rangiranje atributa po važnosti te odbacivanja manje važnih atributa. Također, pokazano je da selekcija atributa može imati različite efekte na konačne rezultate, od zadržavanja točnosti uz smanjenje veličine stabla, povećanja točnosti uz zadržanu veličinu stabla do degradacije performansi modela (ista točnost, ali povećano stablo odlučivanja). To je samo još jedna potvrda činjenice da put do najboljeg modela u dubinskoj analizi nije pravocrtan nego podrazumijeva iterativni proces uz postupno podešavanje parametara modela.

Stabla odlučivanja su se za ovaj model pokazala kao vrlo brz i uspješan algoritam i potvrdila pretpostavku da heuristike ugrađene u dizajn modela mogu izabrati kvalitetan podskup podataka. Iako je jedan od najstarijih algoritama u domeni, stabla odlučivanja u različitim varijantama ostaju privlačna zbog svoje jednostavnosti i mogućnosti da generiraju rezultate koji pomažu istražiteljima interpretirati svojstva skupova podataka koji se analiziraju što je zapravo i krajnji cilj svake dubinske analize.
